% This file is part of the CombinatoricDegeneracies project.
% Copyright 2016 the authors.

% # to-do:
% - draft introduction
% - draft model section
% - perform experiments
% - draft experiment section
% - draft discussion
% - add keywords
% - post on arXiv
% - submit somewhere?

\documentclass[12pt, preprint]{aastex}
\begin{document}

\title{A toy problem for MCMC in the face of labeling degeneracies}
\author{David~W.~Hogg\altaffilmark{1,2,3,4} \& Daniel Foreman-Mackey\altaffilmark{5,6}}
\altaffiltext{1}{Simons Center for Data Analysis, 160 Fifth Avenue, 7th Floor, 
                 New York, NY 10010, USA}
\altaffiltext{2}{Center for Cosmology and Particle Physics, 
                 Department of Physics, New York University, 4 Washington Pl., 
                 room 424, New York, NY, 10003, USA}
\altaffiltext{3}{Center for Data Science, New York University, 726 Broadway, 
                 7th Floor, New York, NY 10003, USA}
\altaffiltext{4}{Max-Planck-Institut f\"ur Astronomie, K\"onigstuhl 17, 
                 D-69117 Heidelberg, Germany}
\altaffiltext{5}{Sagan Fellow}
\altaffiltext{6}{Department of Astronomy, University of Washington, Box 351580, Seattle, WA 98195, USA}

\begin{abstract}
Many inference problems in the natural sciences contain \emph{labeling
  degeneracies} (the label-switching problem or label
non-identifiability):
A model contains some catalog of objects (exoplanets or gravitational
chirps or neuron firings), and the model is invariant to reordering or
re-labeling those objects in the catalog.
This forces the investigator \emph{either} to force a deterministic
ordering---which leads to informative, sharp edges in the prior---or else
directly face a combinatorically large degeneracy in the likelihood
function.
Nonetheless, there are claims in the literature of converged MCMC
samplings and accurate Bayesian evidence integral or fully
marginalized likelihood (FML) calculations, for non-trivial problems
of this type.
Here we present a large family of toy problems that suffer from the
labeling degeneracy but which can be sampled exactly (that is, with
autocorrelation time of unity and without MCMC) and which have
analytic marginalizations (including the FML).
These toy problems can be used to test MCMC (or other) methods for
performing integrations in the face of these degeneracies.
HOGG: We do SOMETHING and find SOMETHING.
It is worthy of note that our toy problem is in general far simpler
than most real inference problems that contain labeling degeneracies.
\end{abstract}

\section{Introduction}

Many problems have combinatoric labeling degeneracies.

You can deal with these by brute force.  Or you can deal with them by
forcing a deterministic ordering.  The first is impossible and the
second is impossible.

Many claims exist that people can sample in these problems.

One interesting point is that you don't necessarily need to sample all
the combinatoric modes; you just need to sample such that all relevant
scientific integrals are okay.

Of course if you want the FML, you are effed.

\section{Model}

\section{Experiments}

\section{Discussion}

\acknowledgments
It is a pleasure to thank
  Alex Barnett (SCDA),
  Brendon J. Brewer (Auckland),
  and
  Jonathan Goodman (NYU)
for valuable discussions and help.

\end{document}
